\addsec{Vorlesung 6 - Lösungen}

\minisec{Übungsaufgabe 1:}

Die minimale Fenstergröße ergibt sich aus dem Bandwidth-Delay-Produkt:
\begin{gather*}
    W >= B * RTT\\
    W >= 100 \frac{Mbit}{s} * 2,4s = 240 Mbit = 30MByte\\
\end{gather*}
Antwort: Die minimale Fenstergröße auf Sender- und Empfängerseite müsste mindestens 30MB betragen.
Bei $W = 65536 Byte$ gilt:
\[B <= \frac{W}{RTT} = \frac{65536 * 8 Bit}{2,4s} = 218,45 \frac{kbit}{s}\]
Antwort: Bei einer Fenstergröße von 65536 Byte wäre die maximale Übertragungsrate $218,45 \frac{kbit}{s}$