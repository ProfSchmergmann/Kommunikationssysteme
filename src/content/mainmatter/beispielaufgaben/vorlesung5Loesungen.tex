\addsec{Vorlesung 5 - Lösungen}

\minisec{Übungsaufgabe 1:}
Die IDs können frei gewählt werden, müssen sich nur unterscheiden:

\begin{flushleft}
    \fbox{%
        \begin{minipage}[c]{0.5\textwidth}
            A -> MTU 2000 -> R
            \centering
            \begin{tabular}{|c|c|c|c|}
                \hline
                ID         & MF         & Total Length  & Offset       \\
                \hline
                \textbf{1} & \textbf{1} & \textbf{1996} & \textbf{0}   \\
                \hline
                \textbf{2} & \textbf{0} & \textbf{1644} & \textbf{247} \\
                \hline
            \end{tabular}
        \end{minipage}}
\end{flushleft}
\fbox{%
    \begin{minipage}[c]{0.5\textwidth}
        R -> MTU 1500 -> B
        \centering
        \begin{tabular}{|c|c|c|c|}
            \hline
            ID          & MF         & Total Length  & Offset       \\
            \hline
            \textbf{11} & \textbf{1} & \textbf{1500} & \textbf{0}   \\
            \hline
            \textbf{12} & \textbf{1} & \textbf{516}  & \textbf{185} \\
            \hline
            \textbf{21} & \textbf{1} & \textbf{1500} & \textbf{247} \\
            \hline
            \textbf{22} & \textbf{0} & \textbf{164}  & \textbf{432} \\
            \hline
        \end{tabular}
    \end{minipage}}
\bigbreak

Auf der ersten Strecke (A -> R) wird das Paket in zwei Unterpakete mit einer Payload-Größe von 1976+1624=3600 Bytes zerlegt.
1976 muss dabei durch 8 teilbar sein!
Auf der zweiten Strecke (R -> B) werden diese zwei Pakete dann in Pakete mit Payload-Größe 1480+496+1480+144=3600 Bytes zerlegt.
1480 muss dabei durch 8 teilbar sein!
Das letzte Paket muss das MF-Bit (\textbf{M}ore \textbf{F}ragments) auf 0 setzen!