\addsec{Vorlesung 5 - Beispielaufgaben}

\minisec{Übungsaufgabe 1:}
Rechner \textbf{A} sei über einen Router \textbf{R} mit Rechner \textbf{B} verbunden.
Für die Strecke von \textbf{A} zum Router gelte \textbf{MTU=2000}, für die Strecke vom Router \textbf{R} zu Rechner \textbf{B} sei die \textbf{MTU 1500}:
Es soll ein Datenpaket der Länge \textbf{3600} Byte übertragen werden (ohne IP-Header!).
Skizzieren Sie den Prozess der Fragmentierung, indem sie die folgenden Tabellen ausfüllen:

\medskip

\begin{minipage}{0.45\textwidth}
    A → MTU 2000 → R
    \centering
    \begin{tabular}{|c|c|c|c|}
        \hline
        ID & MF & Total Length & Offset \\
        \hline
    \end{tabular}
\end{minipage}
\begin{minipage}{0.45\textwidth}
    R → MTU 1500 → B
    \centering
    \begin{tabular}{|c|c|c|c|}
        \hline
        ID & MF & Total Length & Offset \\
        \hline
    \end{tabular}
\end{minipage}