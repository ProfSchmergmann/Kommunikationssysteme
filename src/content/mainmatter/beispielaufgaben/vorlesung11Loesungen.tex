\addsec{Vorlesung 11 - Lösungen}

\minisec{Übungsaufgabe 1:}
Zur Berechnung der CRC Prüfsumme wird das Generatorpolynom in eine Bitfolge mit führender '1' umgewandelt:
\[1x^5 + 0x^4 + 1x^3 + 1x^2 + 0x^1 + 1x^0 ->  101101\]
Das Polynom ist also vom Grad 5, und wir müssen 5 Nullen an die zu sendenden Daten anhängen!
Danach rechnen wir so lange in Modulo-2 Arithmetik (XOR-Verknüpfungen), bis keine Bits mehr übrig sind:

\begin{center}
    \begin{tabular}{c c c c c c c c c c c c c}
        1 & 1 & 1 & 1 & 0 & 1 & 1 & 1 & 0 & 0 & 0 & 0 & 0 \tabularnewline
        1 & 0 & 1 & 1 & 0 & 1 & & & & & & & \tabularnewline
        \hline
        0 & 1 & 0 & 0 & 0 & 0 & 1 & & & & & & \tabularnewline
        & 1 & 0 & 1 & 1 & 0 & 1 & & & & & & \tabularnewline
        \hline
        & 0 & 0 & 1 & 1 & 0 & 0 & 1 & 0 & & & & \tabularnewline
        & & & 1 & 0 & 1 & 1 & 0 & 1 & & & & \tabularnewline
        \hline
        & & & 0 & 1 & 1 & 1 & 1 & 1 & 0 & & & \tabularnewline
        & & & & 1 & 0 & 1 & 1 & 0 & 1 & & & \tabularnewline
        \hline
        & & & & 0 & 1 & 0 & 0 & 1 & 1 & 0 & & \tabularnewline
        & & & & & 1 & 0 & 1 & 1 & 0 & 1 & & \tabularnewline
        \hline
        & & & & & 0 & 0 & 1 & 0 & 1 & 1 & 0 & 0 \tabularnewline
        & & & & & & & 1 & 0 & 1 & 1 & 0 & 1 \tabularnewline
        \hline
        & & & & & & & 0 & \textcolor{blue}{0} & \textcolor{blue}{0} & \textcolor{blue}{0} & \textcolor{blue}{0} & \textcolor{blue}{1} \tabularnewline
    \end{tabular}
\end{center}

Das Ergebnis der Prüfsummenberechnung ist 00001, d.\ h. die zu sendende Bitfolge ist: 11110111\textcolor{blue}{00001}