\addsec{Vorlesung 7 - Beispielaufgaben}

\minisec{Übungsaufgabe 1:}
Beim TCP Verbindungsaufbau wird der 3-way-handshake benutzt.
\begin{enumerate}
    \item Warum wird nicht ein 'einfacher' Handshake, also nur 2 Nachrichten, verwendet?
    \item Wissen beiden Seiten nach der 3. Nachricht sicher, dass die Verbindung aufgebaut ist?
    \item Warum ist der 3-way-handshake bei TCP trotzdem sicher?
\end{enumerate}

\minisec{Übungsaufgabe 2:}
\begin{enumerate}
    \item Erklären Sie das Silly-Window-Syndrom!
    \item Was passiert, wenn eine bestehende TCP-Verbindung über Stunden nicht genutzt wird? Was passiert, wenn einer der beiden verbundenen Rechner abstürzt?
\end{enumerate}

\minisec{Übungsaufgabe 3:}
Woher kennt TCP die Größe der übermittelten Nutzdaten (bzw. wie 'lang' ein übertragenes Segment ist) ?

\minisec{Übungsaufgabe 4:}
Sie möchten mit der West-Küste der USA eine TCP-basierte Kommunikation durchführen.
Damit Ihre Anwendung vernünftig funktioniert benötigen Sie eine Bandbreite von mindestens 40 $\frac{Mbit}{s}$.
Die Round-Trip-Zeit sei 100ms.

Was benötigen Sie im TCP-Protokoll, um dieses Ziel zu erreichen?
Wie genau würde dies ablaufen?