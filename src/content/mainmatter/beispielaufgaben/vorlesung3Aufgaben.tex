\addsec{Vorlesung 3 - Beispielaufgaben}

\minisec{\textbf{Übungsaufgabe 1:}}
Gehen Sie von der folgenden (vereinfacht dargestellten) Routing-Tabelle aus:

\begin{center}
    \begin{tabular}{|c | c|}
        \hline
        Zieleintrag      & Zu nehmender Router \\
        \hline
        139.179.200.0/21 & R1                  \\
        139.179.128.0/18 & R2                  \\
        139.179.112.0/20 & R3                  \\
        139.179.192.0/20 & R4                  \\
        139.179.0.0/16   & R5                  \\
        \hline
    \end{tabular}
\end{center}

An welchen Router werden die folgenden Ziel-IP-Adressen verschickt?

\begin{enumerate}[(a)]
    \item 139.179.60.10
    \item 139.179.210.40
    \item 139.179.197.55
    \item 139.179.205.180
\end{enumerate}

Begründen Sie Ihre Antwort!

\minisec{\textbf{Übungsaufgabe 2:} Gegeben sei die vereinfacht dargestellte Routing-Tabelle:}

\begin{center}
    \begin{tabular}{|c | c|}
        \hline
        Zielnetz (CIDR) & Router \\
        \hline
        128.128.0.0/9   & R1     \\
        128.160.0.0/11  & R2     \\
        128.176.0.0/12  & R1     \\
        128.192.0.0/10  & R1     \\
        default         & R3     \\
        \hline
    \end{tabular}
\end{center}

Können Sie hier Routing-Einträge zusammenfassen?
Falls ja, welche, und wie sehen die zusammengefassten Einträge aus?
Zusatzfrage:
Kann weiter vereinfacht werden, wenn die Präfixe in der Tabelle geändert werden dürfen?