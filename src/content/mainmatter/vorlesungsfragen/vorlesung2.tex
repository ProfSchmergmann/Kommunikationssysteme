\addsec{Vorlesung 2 - Sockets und Darstellungsschicht}

\minisec{ Wer kommuniziert beim Internet-Referenzmodell letztendlich miteinander?}
\begin{itemize}
    \item Die Schichten untereinander und die Schicht 1 mit der Schicht 1
\end{itemize}

\minisec{Welche 3 Arten von 'Adressen' werden im Internet-Referenzmodell für die Kommunikation benötigt?}
\begin{itemize}
    \item Portnummern
    \item IP-Adressen
    \item MAC-Adressen
\end{itemize}

\minisec{Was abstrahiert aus Programmiersicht ein sog. 'Socket'?}
\begin{itemize}
    \item Einen TCP Clienten
\end{itemize}

\minisec{Warum haben Client- und Serversockets eine unterschiedliche Anzahl von Konstruktorparametern?}
\begin{itemize}
    \item Der Clientsocket muss den Zielrechner und den eigenen Port mitgeben, um Daten senden und empfangen zu können, während der Serversocket nur den Port mitgeben muss umd Verbindungen zu akzeptieren.
\end{itemize}

\minisec{Was bewirkt auf Serverseite der Aufruf von accept()?}
\begin{itemize}
    \item Damit wartet der Server auf eingehende Verbindungswünsche.
\end{itemize}

\minisec{Was ist bei der Programmierung von Servern zu beachten, wenn eine hohe Zahl an Anfragen erwartet wird?}
\begin{itemize}
    \item Jeden Clienten in einem eigenen Thread bearbeiten.
\end{itemize}

\minisec{Wodurch unterscheiden sich Little- und Big-Endian-CPUs, und warum kann das zu Problemen bei der Datenkommunikation führen?}
\begin{itemize}
    \item Bei \textbf{Big-Endian} wird das 'große Ende' (also der signifikanteste Wert in der Sequenz) zuerst abgelegt (also in niedrigsten Speicheradresse)
    \item Bei \textbf{Little-Endian} wir der „kleinste Ende“ (also der am wenigsten signifikante Wert) zuerst gespeichert
\end{itemize}

\minisec{Definieren Sie die Begriffe ``Abstrakte Syntax`` und ``Transfersyntax``}
\begin{itemize}
    \item Abstrakte Syntax
    \begin{itemize}
        \item Übertragung der grundsätzlichen Struktur der Daten \[Serialisierung \:- Deserialisierung\]
    \end{itemize}
    \item Transfersyntax
    \begin{itemize}
        \item Übertragung des konkreten Datenstroms mit einer vereinbarten Kodierung \[Marshalling \:- Unmarshalling\]
    \end{itemize}
\end{itemize}

\minisec{Was versteht man unter (De-) Serialisierung?}
\begin{itemize}
    \item Deserialisierung ist die Wiederherstellung eines Objektes ohne Vorwissen über die Typen der Objekte
\end{itemize}

\minisec{Nennen Sie 3 Technologien oder APIs, die das Übertragen von abstrakten/strukturierten Daten über ein Rechnernetz ermöglichen!}
\begin{itemize}
    \item ISO:
    \begin{itemize}
        \item ASN.1 (Abstract Syntax Notation)
    \end{itemize}
    \item Sun ONC (Open Network Computing)-RPC:
    \begin{itemize}
        \item XDR (eXternal Data Representation)
    \end{itemize}
    \item OSF (Open System Foundation)-RPC:
    \begin{itemize}
        \item IDL (Interface Definition Language)
    \end{itemize}
    \item Corba:
    \begin{itemize}
        \item IDL und CDR (Common Data Representation):
        \item CDR bildet IDL-Datentypen in Bytefolgen ab.
    \end{itemize}
    \item W3C:
    \begin{itemize}
        \item XML/SOAP
        \item Darstellung aller Datentypen als (Maschinen-)lesbarer Text. Zu klären: Zeichenkodierung
    \end{itemize}
    \item Java:
    \begin{itemize}
        \item Objektserialisierung, d.h. Abflachung eines (oder mehrerer) Objektes zu einem seriellen Format inkl. Informationen über die Klassen. Deserialisierung ist die Wiederherstellung eines Objektes ohne Vorwissen über die Typen der Objekte
    \end{itemize}
    \item JavaScript:
    \begin{itemize}
        \item JSON (JavaScript Object Notation)
    \end{itemize}
\end{itemize}

\minisec{Erklären Sie den Aufbau und die syntaktischen Elemente eines XML-Dokumentes!}
\begin{itemize}
    \item XML-Dokumente: Zeichendaten und Auszeichnungen
    \begin{itemize}
        \item XML- und Dokumenttyp-Deklarationen
        \lstset{language=XML}
        \begin{lstlisting}[label={lst:lstlisting}]
            <?xml version=``1.0`` encoding=``UTF-8``?>
        \end{lstlisting}
        \item Elemente mit möglichem Inhalt \lstset{language=XML}
        \begin{lstlisting}[label={lst:lstlisting2}]
            <name>Volker Sander</name>
        \end{lstlisting}
        \item Attributen in Elementen
        \lstset{language=XML}
        \begin{lstlisting}[label={lst:lstlisting3}]
            <name attribute=``Wert``>
        \end{lstlisting}
        \item Entity-Referenzen (\&lt; statt <)
        \item Kommentare \linebreak
        <!– Das ist ein Kommentar -->
        \item Processing Instructions – werden an die aufrufende Instanz weitergeleitet \linebreak
        <?name pidata?>
        \item Jedes Dokument entspricht einer Baumstruktur (DOM – Document Object Model)
        \item Die Baumknoten werden Elemente genannt
        \item Es kann nur ein Element an der Wurzel geben
        \item Syntaxregeln müssen strikt eingehalten werden
        \begin{itemize}
            \item Jedes Starttag muss ein Endtag haben
            \lstset{language=XML}
            \begin{lstlisting}[label={lst:lstlisting4}]
                <BOOK>...</BOOK>
                <BOOK />
            \end{lstlisting}
            \item Verschachtelung nur mit vollständigen Tags möglich:
            \lstset{language=XML, breaklines=true}
            \begin{lstlisting}[label={lst:lstlisting5}]
                <BOOK> <LINE> This is OK </LINE> </BOOK>
            \end{lstlisting}
            \lstset{language=XML, breaklines=true}
            \begin{lstlisting}[label={lst:lstlisting6}]
                <LINE> <BOOK> This is </LINE> definitely NOT </BOOK> OK
            \end{lstlisting}
            \item Attributwerte müssen mit ‘ oder “ abgegrenzt werden
        \end{itemize}
    \end{itemize}
\end{itemize}

\minisec{Was ist der Unterschied zwischen Wohlgeformtheit und Validität bei einem XML-Dokument?}
\begin{itemize}
    \item Wohlgeformtheit:
    \begin{itemize}
        \item Dokument entspricht den syntaktischen Regeln
        \begin{itemize}
            \item Genau ein Dokument-Element
            \item Jedes öffnende Tag hat ein schließendes Tag.
            \item Die Verschachtelung ist balanciert.
        \end{itemize}
    \end{itemize}
    \item Validät:
    \begin{itemize}
        \item Dokument ist:
        \begin{itemize}
            \item wohlgeformt
            \item konform zu einer fest vorgegebenen \underline{Dokumentenstruktur}
            \item Document Type Description (DTD)
            \item XML Schema Definition (XSD)
        \end{itemize}
    \end{itemize}
\end{itemize}

\minisec{Wozu dient ein XML-Schema?}
\begin{itemize}
    \item Ein XML Schema enthält in XML notierte Regeln, die erlaubte Elementbezeichner, Reihenfolgen, Inhalte und Attribute mit Wertebereichen aufzählen
\end{itemize}

\minisec{Nennen Sie 3 Kompositoren (und deren Ergänzungen), die in XML-Schemas verwendet werden können!}
\begin{itemize}
    \item
    \lstset{language=XML, breaklines=true}
    \begin{lstlisting}[label={lst:lstlisting7}]
    <xsd:sequence>
        [Inhalt 1]
        ...
        [Inhalt n]
    </xsd:sequence>
    \end{lstlisting}
    \item
    \lstset{language=XML, breaklines=true}
    \begin{lstlisting}[label={lst:lstlisting8}]
    <xsd:all>
        [Elementdeklarationen]
    </xsd:all>
    \end{lstlisting}
    \item
    \lstset{language=XML, breaklines=true}
    \begin{lstlisting}[label={lst:lstlisting9}]
    <xsd:choice>
        [Inhalt a]
        ...
        [Inhalt x]
    </xsd:choice>
    \end{lstlisting}
\end{itemize}

\minisec{Wozu dient das SOAP-Protokoll, und wie ist es grundlegend aufgebaut?}
\begin{itemize}
    \item SOAP (ursprünglich für Simple Object Access Protocol) ist ein Netzwerkprotokoll, mit dessen Hilfe Daten zwischen Systemen ausgetauscht und Remote Procedure Calls durchgeführt werden können.
    \item SOAP regelt, wie Daten in der Nachricht abzubilden und zu interpretieren sind
    \item SOAP stellt so eine Konvention für entfernte Prozedur-/Methodenaufrufe dar
    \item Eine SOAP-Nachricht besteht zunächst auf der obersten Ebene aus einem Envelope
    \item Der Envelope enthält einen optionalen Header sowie die eigentliche Nachricht im SOAP Body
\end{itemize}

\minisec{Wie ist eine JSON-Datei grundsätzlich aufgebaut?}
\begin{itemize}
    \item Beispiel:
    \begin{lstlisting}[label={lst:lstlisting10}]
        {
            "id" : 564146,
            "Name" : "Mustermann",
            "Vorname" : "Max"
        }
    \end{lstlisting}
\end{itemize}