\addsec{Vorlesung 9 - Anwendungsprotokolle}

\minisec{Was bedeuten die Abkürzungen HTTP und HTML? Was ist ein 'HT' ?}
\begin{itemize}
    \item \underline{HTML:} Sprache zur Beschreibung der Seiten
    \item \underline{HTTP:} Protokoll zur Übertragung der Seiten
    \item \underline{'HT':} \todo A
\end{itemize}

\minisec{Erklären Sie den Zusammenhang zwischen URI, URL und URN!}
\begin{itemize}
    \item \underline{\textbf{URL: }} Spezifikation von Ort und Zugriffsmodalitäten
    \item \underline{\textbf{URL:} (\textbf{U}niform \textbf{R}esource \textbf{L}ocator)} Adressierung von Informationsobjekten mit Festlegung des Zugangs-Protokolls (Ort der Ressource). RFC2141
    \item \underline{\textbf{URN:} (\textbf{U}niform \textbf{R}esource \textbf{N}ame)} Adressierung von Objekten ohne ein Protokoll festzulegen (Eindeutige und gleichbleibende Referenz – Name der Ressource). RFC1738
    \item \textbf{URI} $=$ \textbf{URL} $\cup$ \textbf{URN}
\end{itemize}

\minisec{Wie ist ein HTTP-Request-Header aufgebaut? Wie wird der Header von den Daten getrennt, und woher weiß man die Länge der Daten?}
\begin{itemize}
    \item \todo A
\end{itemize}

\minisec{Welche Gruppen von HTTP Status-Codes kennen Sie?}
\begin{itemize}
    \item
\end{itemize}

\minisec{Welche HTTP-Methoden existieren neben 'GET'?}
\begin{itemize}
    \item \textbf{HEAD}
    \item \textbf{POST}
    \item \textbf{PUT}
    \item \textbf{DELETE}
\end{itemize}

\minisec{Was ist MIME? Welche Attribute (mind. 3) im Header werden dazu verwendet, und was bedeuten (bzw. definieren) sie jeweils?}
\begin{itemize}
    \item \textbf{MIME} = \textbf{M}ultipurpose \textbf{I}nternet \textbf{M}ail \textbf{E}xtensions
    \item Attribute:
    \begin{itemize}
        \item \textcolor{blue}{Content-Type:} \textbf{IMAGE/JPEG}; name="picture.jpg"
        \item \textcolor{blue}{Content-Transfer-Encoding:} \textbf{BASE64}
        \item \textcolor{blue}{Content-ID:} <PINE.LNX.3.91.960212212235.325B@localhost>
    \end{itemize}
\end{itemize}

\minisec{Erklären Sie das Konzept von 'Virtual Hosts' bei HTTP!}
\begin{itemize}
    \item Auf einem Rechner sollen verschiedene Domains und Web-Server zur Verfügung stehen → Jeder Server hat die gleiche IP, aber ggf. unterschiedliche DNS-Namen!
    \item Ein oder mehrere Webserver (Software) sollen die Anfragen, für die auf dem Rechner vorhandenen Domains, beantworten
    \item Typische Anwendung: Web-Hosting (Provider)
\end{itemize}

\minisec{Wo ist der SSL-Layer im ISO/OSI oder Internet-Schichtenmodell angesiedelt?}
\begin{itemize}
    \item In der Darstellungsschicht (Schicht 6)
\end{itemize}

\minisec{Erklären Sie die Unterschiede/Vorteile/Nachteile von symmetrischer und asymmetrischer Verschlüsselung!}
\begin{itemize}
    \item \todo A
\end{itemize}

\minisec{Was ist ein Zertifikat? Wer stellt es aus, und welche Informationen enthält es?}
\begin{itemize}
    \item \todo A
\end{itemize}

\minisec{Welche Eigenschaften hat eine Hash-Funktion? Wo wird Sie im Kontext der Verschlüsselung eingesetzt?}
\begin{itemize}
    \item \todo A
\end{itemize}

\minisec{Wie kann man bei 2 Kommunikationspartnern ein 'Geheimnis' (z.B. einen Schlüssel zur symmetrischen Verschlüsselung) erzeugen, ohne dieses Geheimnis über das Netzwerk oder einen anderen Kanal auszutauschen?}
\begin{itemize}
    \item \todo A
\end{itemize}

\minisec{Welche Aufgaben hat das SSL Handshake Protokoll und das SSL Record Protokoll?}
\begin{itemize}
    \item SSL Handshake Protokoll:
    \begin{itemize}
        \item Den stärksten gemeinsam unterstützten Algorithmus ermitteln
        \item Authentifikation der Kommunikationspartner (Client optional)
        \item Ermitteln eines Session Keys zur symmetrischen Verschlüsselung (optional)
    \end{itemize}
    \item SSL Record Layer:
    \begin{itemize}
        \item Vollständig getrennt vom Handshake Protokoll
        \item Verschickt Daten symmetrisch mit dem im Handshake ausgehandelten Verschlüsselungsalgorithmen und Session Keys
        \item Bildet zu jedem Datenblock einen Message Digest zur Sicherung der Integrität
    \end{itemize}
\end{itemize}

\minisec{Warum sollte 'einfaches' FTP heute nicht mehr verwendet werden?}
\begin{itemize}
    \item \todo A
\end{itemize}

\minisec{Warum gibt es in einem typischen Heimnetzwerk Probleme mit dem FTP 'Active' Mode?}
\begin{itemize}
    \item \todo A
\end{itemize}

\minisec{Informieren Sie sich über die Entstehungsgeschichte und Funktionalität von SSL und SSH!}
\begin{itemize}
    \item \todo A
\end{itemize}

\minisec{Wie funktionieren SFTP und TFTP?}
\begin{itemize}
    \item \todo A
\end{itemize}

\minisec{Welches Protokoll wird zum Versenden von Emails verwendet? Was passiert konkret bei Versenden, und wie ist das DNS beteiligt?}
\begin{itemize}
    \item \todo A
\end{itemize}

\minisec{Welche Protokolle werden zum Abrufen von Emails verwendet, und wie unterscheiden sie sich?}
\begin{itemize}
    \item \textcolor{blue}{Simple Mail Transfer Protocol (SMTP):}
    \begin{itemize}
        \item Versenden von eMails über TCP-Verbindung (Port 25)
        \item SMTP ist ein einfaches ASCII-Protokoll
        \item Ohne Prüfsummen, ohne Verschlüsselung
        \item Ist der Server zum Empfangen bereit, signalisiert er dies dem Client. Dieser sendet die Information, von wem die eMail kommt und wer der Empfänger ist. Ist der Empfänger dem Server bekannt, sendet der Client die Nachricht, der Server bestätigt den Empfang.
    \end{itemize}
    \item \textcolor{blue}{Post Office Protocol Version 3 (POP3):}
    \begin{itemize}
        \item Abholen der eMails beim Server über eine TCP-Verbindung, Port 110
        \item Befehle zum An- und Abmelden, Nachrichten herunterladen, Nachrichten auf dem Server löschen oder liegen lassen, Nachrichten ohne vorherige Übertragung vom Server direkt löschen
    \end{itemize}
    \item \textcolor{blue}{IMAP (Interactive Mail Access Protocol):}
    \begin{itemize}
        \item Hier werden die eMails nicht abgerufen und lokal gespeichert, sondern bleiben auf dem Server liegen!
    \end{itemize}
\end{itemize}

\minisec{Warum ist einfaches SMTP unsicher?}
\begin{itemize}
    \item \todo A
\end{itemize}

\minisec{Wozu dient das (veraltete) TELNET-Protokoll? Wozu kann es (z.B. im Praktikum) sinnvoll eingesetzt werden?}
\begin{itemize}
    \item TCP ermöglicht den transparenten, interaktiven Gebrauch von „entfernten“ Maschinen
    \item verbreitetes Protokoll: TELNET, welches auf einer Client/Server-Kommunikation basiert
    \item Ein „Pseudo-Terminal“ des Servers interpretiert Zeichen, als kämen sie von der eigenen Tastatur
    \item bei Antwort des Servers umgekehrter Weg (Pseudo-Teminal fängt Antwort ab, leitet sie über TCP an den Client weiter, der die Ausgabe am Bildschirm macht
    \item \textbf{Benutzername und Passwort werden unverschlüsselt übertragen}
\end{itemize}

\minisec{Welches Protokoll sollte heute zum Login auf entfernten Rechnern verwendet werden?}
\begin{itemize}
    \item \textcolor{blue}{ssh} adressiert die Sicherheitsprobleme von telnet und rlogin.
    Es ist ein Protokoll zur Erstellung einer sicheren Verbindung zwischen zwei Systemen.
    Alle während der Verbindung gesendeten und empfangenen Daten werden mit einer 128 Bit-Verschlüsselung verschlüsselt.
    \item \textcolor{blue}{ssh} unterstützt verschiedene Authentisierungsarten:
    \begin{itemize}
        \item Bei der so genannten hostbased-Authentifizierung akzeptiert ein Rechner ohne eigene account-spezifische Tests die Vorgaben eines fremden Rechners. Es wird höchstens die Identität des fremden Rechners überprüft.
        \item Die Authentifikation mit einem Passwort ist derzeit die "übliche" Methode, um sich an einem Rechner anzumelden. Die Sicherheit dieses Mechanismus beruht auf der Geheimhaltung des Passwortes, dessen Übertragung allerdings verschlüsselt wird
        \item Um auch das Übertragen eines verschlüsselten Passwortes zu vermeiden, werden die so genannten public-key-Verfahren eingesetzt
    \end{itemize}
\end{itemize}

\minisec{Wie kann mein einen sicheren 'Tunnel' zu/von einem beliebigen (TCP-) Port einrichten?}
\begin{itemize}
    \item Mit \textcolor{blue}{SSH: Port-Forwarding:}
    \begin{itemize}
        \item verschlüsselte Verbindung zwischen zwei beliebigen Ports
        \item kann auch ohne Shell genutzt werden
        \item lokaler Port führt direkt auf den Zielport, als wäre dieser lokal
    \end{itemize}
\end{itemize}

\minisec{Wozu wird SNMP verwendet? Welches Transportprotokoll verwendet es? Was ist ein SNMP-Agent und eine MIB?}
\begin{itemize}
    \item Transportprotokoll: UDP
    \item SNMP: Protokoll, das festlegt, wie Management-Information kommuniziert wird (Formate und Bedeutung von SNMP-Nachrichten)
    \item MIB (Management Information Base): Die MIB spezifiziert die Informationseinheiten (items), die vorgehalten werden müssen, und welche Operationen darauf erlaubt sind.
    \item \todo 9 Wofür wird SNMP verwendet?
\end{itemize}