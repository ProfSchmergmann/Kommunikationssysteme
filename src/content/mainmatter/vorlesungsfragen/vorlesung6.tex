\addsec{Vorlesung 6 - Send and Wait, Sliding Window}

\minisec{Nennen Sie die 2 wichtigen Transportprotokolle im Internet! Wie unterscheiden sie sich?}
\begin{itemize}
    \item Transmission Control Protocol (TCP) -> zuverlässig
    \begin{itemize}
        \item Kommunikationsprimitive, Sockets, Ports
        \item Virtuelle Verbindungsorientierung: Modell zweier Streams die die Prozesse miteinander verbinden;
        „Open“-“Close“ der Streams sind die virtuellen Verbindungen
        \item Flusskontrolle, Staukontrolle
    \end{itemize}
    \item User Datagram Protocol (UDP) -> unzuverlässig
    \begin{itemize}
        \item Kommunikationsprimitive, Sockets, Ports
        \item Postkartenmodell: Ein Empfangspunkt (Port) für alle Nachrichten, versenden ohne Absprache mit dem Empfänger
    \end{itemize}
\end{itemize}

\minisec{Was bedeutet eine 'virtuelle Verbindungsorientierung'?}
\begin{itemize}
    \item Modell zweier Streams die die Prozesse miteinander verbinden;
    „Open“-“Close“ der Streams sind die virtuellen Verbindungen
\end{itemize}

\minisec{Erklären Sie das ARQ-Protokoll und die Randbedingungen, unter denen dieses Protokoll eine sichere Datenübertragung gewährleistet! Wo und warum werden hier Timeouts eingesetzt?}
\begin{itemize}
    \item Segmentweises Vorgehen
    \item Timer zur Fehleridentifikation
    \begin{itemize}
        \item Verschicke ein Segment und warte auf den Empfang des ACKs
        \item Bei Empfang eines ACKs wird das nächste Segment verschickt, andernfalls wird die Übertragung wiederholt
        \item Jeder Datenstrom wird in Segmente unterteilt, z.B. 1460 Bytes Nutzdaten
        \item Nach jedem gesendeten Segment wird auf eine Bestätigung gewartet;
        während der Wartezeit werden keine weiteren Segmente übertragen
    \end{itemize}
\end{itemize}

\minisec{Wie ist der Nutzungsgrad einer Verbindung definiert? Warum gibt es bei Verbindungen mit hoher RTT hier Probleme?}
\begin{itemize}
    \item \underline{Nutzungsgrad:} Verhältnis von genutzter Übertragungszeit zu gesamter Übertragungsdauer (inkl. der Wartezeit)
    \item \[\rho = \frac{\textrm{genutzte Uebertragungszeit}}{\textrm{genutzte Uebertragungszeit} + \textrm{Wartezeit}}\]
\end{itemize}

\minisec{Was bedeutet 'Pipelining' im Kontext der Datenübertragung? Was bedeutet in diesem Zusammenhang 'Fensterbreite'? Welche Schwierigkeiten entstehen bei Fehlerfällen?}
\begin{itemize}
    \item Pipelining:
    \begin{itemize}
        \item Senden ohne vorheriges ACK
        \item \underline{Fensterbreite beim Sender:} Anzahl der Segmente, die ohne Bestätigung gesendet werden
        \item \underline{Fensterbreite beim Empfänger:} Anzahl der Segmente, die auch bei Lücken oder Fehlern zwischengespeichert werden
    \end{itemize}
    \item Probleme beim Empfänger:
    \begin{itemize}
        \item Empfang eines out-of-order-Segments
        \begin{itemize}
            \item Eigentlich wurde ein anderes Segment erwartet
            \item Je nach Implementierung wird das nicht erwartete Segment einfach ignoriert, es wird dann verworfen
        \end{itemize}
        \item Empfang eines fehlerbehafteten Segments
        \begin{itemize}
            \item Keine Reaktion (Der Sender wird irgendwann eine erneute Übertragung vornehmen, da ja keine Bestätigung verschickt wurde)
            \item Falls unterstützt: Verschicken einer „negativen“ Bestätigung oder einer Nachricht, die dies kommuniziert (als Indiz für einen möglichen Paketverlust)
        \end{itemize}
        \item Probleme beim Sender:
        \begin{itemize}
            \item Timeout -> Worauf bezieht sich dieser?
            \item Empfang einer nicht erwarteten Bestätigung (out-of-order) ACKs
            \item Empfang einer negativen Bestätigung (falls es die gibt)
        \end{itemize}
    \end{itemize}
\end{itemize}

\minisec{Wie löst das Go-Back-N Verfahren Übertragungsfehler?}
\begin{itemize}
    \item Der Empfänger sendet nach dem Fehler kein ACK mehr
    \item Der Sender läuft dadurch in ein Timeout und fängt wieder ein Segment nach dem letzten bestätigten Segment wieder an
    \item Alle direkt nach dem Fehler versendeten Segmente werden verworfen
\end{itemize}

\minisec{Wie groß ist die Fensterbreite beim Sender und Empfänger bei 'Go-Back-N'?}
\begin{itemize}
    \item Fensterbreite Sender: > 1
    \item Fensterbreite Empfänger: 1
\end{itemize}

\minisec{Warum sollten sich bei 'Sliding-Window' Verfahren Sender und Empfänger bzgl. ihrer Fensterbreiten koordinieren?}
\begin{itemize}
    \item \todo A
\end{itemize}

\minisec{Was passiert, wenn das Sendefenster größer als das Empfangsfenster ist?}
\begin{itemize}
    \item \todo A
\end{itemize}

\minisec{Was passiert, wenn das Empfangsfenster größer als das Sendefenster ist?}
\begin{itemize}
    \item \todo A
\end{itemize}

\minisec{Was passiert, wenn beide Fensterbreiten 1 sind?}
\begin{itemize}
    \item \todo A
\end{itemize}

\minisec{Warum sind große Fensterbreiten häufig schwer zu realisieren?}
\begin{itemize}
    \item \todo A
\end{itemize}

\minisec{Wie kann der Empfänger eine 'Flusskontrolle' bei der Datenübertragung realisieren?}
\begin{itemize}
    \item \todo A
\end{itemize}

\minisec{Leiten Sie die Berechnung der benötigten Fensterbreite aus dem 'Bandwidth-Delay'-Produkt her!}
\begin{itemize}
    \item \todo A
\end{itemize}

\minisec{Wie funktioniert das "Selective Repeat"-Verfahren? Werden hier auf Empfängerseite Segmente verworfen im Fehlerfall?}
\begin{itemize}
    \item Mittelbare Kenntnis über etwaig bereits empfangene Segmente, Empfänger puffert nicht erwartete Segmente, so dass sich bei Neuübertragung der „Lücke“ das Fenster ruckartig verschiebt
    \item \todo Erklärung
\end{itemize}

\minisec{Wie funktioniert das "Selective Reject"-Verfahren? Werden hier auf Empfängerseite Segmente verworfen im Fehlerfall?}
\begin{itemize}
    \item Unmittelbare Kenntnis über unerwartete Ereignisse und sofortige Reaktion zur Behebung
    \item \todo Erklärung
\end{itemize}

\minisec{Was bedeutet eine kumulative Bestätigung?}
\begin{itemize}
    \item Eine Bestätigung zeigt an, dass alle Daten bis zu der verwendeten Nummer/Byte Position korrekt empfangen wurden
    \item Kommt ein Out-of-Order-Segment an, so wird demnach eine Bestätigung verschickt, die schon einmal verschickt wurde (“ich warte immer noch auf x“)
    \item Speichert der Empfänger außerhalb der Reihenfolge empfangene Segmente zwischen, dann bedeutet ein „Lückenschluss“, dass dieser einen „Sprung“ bei den Bestätigungen bedingt
\end{itemize}

\minisec{Wie können mehrfache ACKs auch als NACK gewertet werden?}
\begin{itemize}
    \item Kommt ein Out-of-Order-Segment an, so wird demnach eine Bestätigung verschickt, die schon einmal verschickt wurde (“ich warte immer noch auf x“)
    \item Wenn ein Segment verloren gegangen ist, dann werden alle weiteren empfangenen Segmente die gleiche Position anzeigen
    \item Duplicate Acknowledgments: Empfang einer Bestätigung, die eine schon versendetes Feeback erneut gibt: (“ich warte immer noch auf x“)
    \item Segmente können sich überholen, deshalb ist das nicht sofort als NACK aufzufassen
    \item \underline{Aber:} Triple Duplicate Acknowledgement wird als NACK aufgefasst!
\end{itemize}