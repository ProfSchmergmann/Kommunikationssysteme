\addsec{Vorlesung 5 - ARP, ICMP und IPv6}

\minisec{Welche unterschiedlichen Aufgaben und Eigenschaften hat eine IP- gegenüber einer MAC-Adresse?}
\begin{itemize}
    \item IP-Adresse: ‚Logische‘ Adressierung (Layer 3)
    \item MAC-Adresse: ‚Physikalische Adressierung‘ (Layer 2)
\end{itemize}

\minisec{Wozu wird das ARP-Protokoll eingesetzt? In welchem 'Layer' wird dieses Protokoll implementiert?}
\begin{itemize}
    \item \textcolor{blue}{ARP} (Address Resolution Protocol) Layer-2 Protokoll:
    \begin{itemize}
        \item Ermöglicht das Herausfinden einer MAC-Adresse auf Basis einer versendeten IP-Adresse
        \item Funktioniert über Broadcast-Paket an MAC FF:FF:FF:FF:FF:FF
        \item Der Host mit der angefragten IP-Adresse antwortet mit seiner MAC-Adresse
        \item Der empfangende Knoten ‚cached‘ i.d.R. die MAC-Adressen
    \end{itemize}
\end{itemize}

\minisec{Wo befindet sich die ARP-Tabelle?}
\begin{itemize}
    \item Das Address Resolution Protocol (ARP) arbeitet auf der Schicht 2, der Sicherungsschicht, des OSI-Schichtenmodells
\end{itemize}

\minisec{Welche Vorteile hat das DHCP-Protokoll gegenüber dem älteren RARP-Protokoll?}
\begin{itemize}
    \item Probleme von RARP
    \begin{itemize}
        \item Ethernet-Broadcasts sind auf Subnetze beschränkt.
        In einem LAN mit Subnetze braucht man mehrere RARP-Server
        \item Durch RARP erfährt ein Rechner nur seine IP-Adresse!
        Zu einer vollständigen Konfiguration einer Netzwerkschnittstelle gehören noch mindestens Netzmaske und Default-Gateway.
        \item DHCP hat RARP heute komplett abgelöst!
    \end{itemize}
    \item \textcolor{blue}{DHCP} (\textcolor{blue}{D}ynamic \textcolor{blue}{H}ost \textcolor{blue}{C}onfiguration \textcolor{blue}{P}rotokoll)
    \begin{itemize}
        \item Basiert auf dem älteren BOOTP-Protokoll und ist zu diesem (eingeschränkt) kompatibel
        \item Realisiert als Application Protokoll über UDP Port 67
        \item Ermöglicht die Konfiguration über Subsystemgrenzen (DHCP Relay Agents) und mit Szenarien mit mehreren DHCP-Servern
        \item Erlaubt das Konfigurieren aller wichtigen Parameter (IP/Mask/default Gateway/DNS Server)
        \item Erlaubt das zeitlich u.A. beschränkte Vergeben von IPKonfigurationen (leases) und das Steuern der Vergabe
        \item Sicherheitsprobleme: MAC-Spoofing, DHCP Starvation etc.
    \end{itemize}
\end{itemize}

\minisec{Erklären Sie die Funktion der Felder ``Time to live`` und ``Fragment offset`` im IPv4 Header!}
\begin{itemize}
    \item \textbf{TTL (Time-to-Live): } Mit TTL gibt der Sender die Lebensdauer des Pakets in Sekunden an.
    Jede Station, die ein IP-Paket weiterleiten muss, zieht von diesem Wert mindestens 1 ab. Hat der TTL-Wert 0 erreicht, wird das IP-Paket verworfen. Dieser Mechanismus verhindert, dass Pakete ewig Leben, wenn sie nicht zustellbar sind. TTL-Werte zwischen 30 und 64 sind typisch.
    \item \textbf{Fragment-Offset:} Enthält ein IP-Paket fragmentierte Nutzdaten, steht in diesem Feld die Position der Daten im ursprünglichen IP-Paket.
\end{itemize}

\minisec{Warum werden IP-Pakete auf Ihrem Weg ggf. fragmentiert? Welche Rolle spielt dabei die MTU?}
\begin{itemize}
    \item IP-Pakete sind gegebenenfalls zu groß und werden daher fragmentiert.
    \item Die \textcolor{blue}{MTU} (\textcolor{blue}{M}aximum \textcolor{blue}{T}ransfer \textcolor{blue}{U}nit) beschreibt die maximale Größe und liegt auf Layer 2.
\end{itemize}

\minisec{Wie funktioniert das Verfahren der ``Path MTU Discovery``?}
\begin{itemize}
    \item Das Quell-System schickt die Pakete mit der Flagge DF (don‘t fragment)
    \item Wenn ein Router eine zu kleine MTU erkennt, wird das Paket verworfen und eine \textcolor{blue}{ICMP}-Nachricht zurück gesendet
    \item Das Quell-System kann jetzt kleinere Pakete erzeugen, die weiter geleitet werden können
    \item Der Prozess muss ggf. mehrmals wiederholt werden!
\end{itemize}

\minisec{Welche Aufgaben hat das ICMP-Protokoll?}
\begin{itemize}
    \item Das \textcolor{blue}{I}nternet \textcolor{blue}{C}ontrol \textcolor{blue}{M}essage \textcolor{blue}{P}rotocol ist ein Steuerprotokoll der Schicht 3, welches auf IP aufbaut! Dieses Protokoll wird z.B. von Routern verwendet, wenn etwas Unerwartetes passiert.
    \item Aufgaben:
    \begin{itemize}
        \item Mitteilung von Problemen beim Paketversand
        \item Echo-Anfragen (existiert der Zielknoten?)
        \item Unterstützung von höheren Protokollen und Anwendungen (z. B. Path MTU discovery, traceroute, \ldots)
    \end{itemize}
\end{itemize}

\minisec{Wie lang (in Bytes) ist eine IPv6-Adresse? Nennen Sie 4 Vorteile von IPv6 gegenüber IPv4!}
\begin{itemize}
    \item 128-Bit-Adressen (8 Gruppen zu je 4 Hexadezimal-Zahlen)
    \item Vorteile von IPv6 gegenüber IPv4:
    \begin{itemize}
        \item längere Adressen und dadurch ein größerer Adressraum
        \item mehrere IPv6-Adressen pro Host mit unterschiedlichen Gültigkeitsbereichen
        \item Multicast durch spezielle Adressen
        \item schnelleres Routing
    \end{itemize}
\end{itemize}