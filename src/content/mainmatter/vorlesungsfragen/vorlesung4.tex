\addsec{Vorlesung 4 - Routingprotokolle und NAT}

\minisec{Erklären Sie den Unterschied zwischen Routing und Forwarding. Ordnen Sie den beiden Begriffen die Begriffe Control Plane und Data Plane zu.}
\begin{itemize}
    \item Das eigentliche Routing (die Control-Plane) ist verantwortlich für Wegewahl, mit der die Ende-zu-Ende-Kommunikation erreicht wird
    \item Das Weiterleiten (engl.\ forwarding) der Pakete (die Data-Plane) gemäß den Vorgaben der Control-Plane
\end{itemize}

\minisec{Mit welchem Algorithmus kann ein optimaler Quelle-Senke Baum berechnet werden?}
Mit dem Dijkstra Algorithmus

\minisec{Welche 2 Arten von Routingverfahren kennen Sie, und welche Unterschiede existieren zwischen den Verfahren?}
\begin{itemize}
    \item Distanzvektoralgorithmen: ohne Kenntnis der gesamten Netztopologie
    \item Link-State Routing: mit Kenntnis der gesamten (zumindest im AS) vorhandenen Netztopologie
\end{itemize}

\minisec{Erklären Sie den Bellmann-Ford Algorithmus mit eigenen Worten}
\begin{itemize}
    \item Definiere:
    \begin{itemize}
        \item \(c_x(v) \coloneqq\) Kosten zum Nachbarn v von x aus
        \item \(d_x(y) \coloneqq\) Kosten des günstigsten Weges von x nach y
    \end{itemize}
    \item Gibt es einen Weg zwischen x und y und sind x und y nicht direkt verbunden, dann muss es einen Nachbarn v von x geben für den gilt:
    \[d_x(y) = \min \{c_x(v) + d_v(y) \}\]
    hierbei wird das Minimum über alle Nachbarn von x genommen.
    Die Kosten der direkten Verbindung wird als bekannt angenommen
\end{itemize}

\minisec{Erklären Sie wie es zum sog. Count to Infinity-Problem kommen kann!}
\begin{itemize}
    \item Der Ausfall von Routen/Routern führt zum „Count-to-Infinity“ Problem: \[A -\: B -\: C\]
    \item Betrachte B: Router A fällt aus, bekommt aber von C gesagt, dass er einen Weg nach A kennt (der allerdings über B führt, was B nicht weiß \ldots)
    → Die Kosten zum Weg nach A schaukeln sich langsam auf \ldots
\end{itemize}

\minisec{Wie funktioniert das Link-State Routing?}
\begin{itemize}
    \item Jeder Router führt die folgenden Schritte durch:
    \begin{enumerate}
        \item Die Nachbarn und deren Netzadressen ermitteln
        \item Die Kosten zu jedem seiner Nachbarn festlegen
        \item Ein Paket zusammenstellen, in dem alles steht, was bisher gelernt wurde
        \item Dieses Paket an alle anderen Router senden und von allen anderen Routern derartige Pakete empfangen
        \item Den kürzesten Pfad zu allen anderen Routern berechnen
    \end{enumerate}
    \item Dadurch wird die gesamte Topologie in jedem Router abgebildet.
    Anwendung des Dijkstra-Algorithmus möglich!
\end{itemize}

\minisec{Nennen Sie jeweils ein konkretes Routing-Protokoll, was das Distanzvektor-Verfahren bzw. das Link-State-Verfahren benutzt.}
\begin{itemize}
    \item RIP, RIP2: Intra-AS, Distance-Vector, veraltet
    \item IS-IS: Intra-AS, ursprünglich für DECnet, danach ISO Standard und Grundlage für OSPF, (Shortest Path First)
    \item OSPF: Intra-AS, Link-State, Unicast (Shortest Path First)
    \item BGP: Border Gateway Protokol Inter-AS, Distance-Vector
\end{itemize}

\minisec{Warum werden Distanzvektor-Verfahren gerne zwischen autonomen Systemen eingesetzt?}
\begin{itemize}
    \item Leicht zu implementieren, einfache Berechnung
    \item Neue, bessere Routen werden im Netz schnell propagiert
\end{itemize}

\minisec{Was sind ''private'' IP-Netze und -Adressen?}
\begin{itemize}
    \item Jeder Endkunde (jede kleinere Firma etc.) bekommt lediglich nur eine eindeutige öffentliche IPv4 Adresse vom ISP, d.\ h. es existiert auch nur ein Zugangspunkt (Gateway) zu diesem Netz
    \item Innerhalb des Hauses/der Firma wird ein ``privates`` Netz betrieben, was die spezifizierten privaten IP-Adressblöcke benutzt:
    \begin{itemize}
        \item \( 10.0.0.0 -\: 10.255.255.255 \)
        \item \( 172.16.0.0 -\: 172.31.255.255 \)
        \item \( 192.168.0.0 -\: 192.168.255.255 \)
    \end{itemize}
\end{itemize}

\minisec{Wie werden solche Adressen im Zusammenhang mit NAT eingesetzt?}
Der NAT-Router ersetzt die Source-IP (z. B. 10.0.0.3) mit der öffentlichen Adresse

\minisec{Was passiert beim Verschicken eines IP-Paketes über einen NAT-Router?}
Der NAT-Router ersetzt die Source-IP (z. B. 10.0.0.3) mit der öffentlichen Adresse

\minisec{Was passiert anschließend beim Empfangen eines Antwortpaketes (von einem Server außerhalb des privaten Netzes)?}
Der Server adressiert in seiner Antwort ein Ziel-Port, das der NAT-Router dann wieder in die interne IP und die interne Port-Nummer zurück übersetzt.

\minisec{Warum verletzt das NAT-Verfahren die Schichtenarchitektur?}
Layer 3 kennt Details des Layer 4, was eigentlich nicht sein darf.

\minisec{Nennen Sie 2 Szenarien, wo der Einsatz von NAT zu Problemen führen würde.
Begründen Sie jeweils ihre Antwort!}
\begin{itemize}
    \item Es wird nicht TCP/UDP verwendet
    \begin{itemize}
        \item Begründung: Wenn TCP/UDP nicht verwendet wird, existiert ggf.\ das Konzept der Portnummern überhaupt nicht mehr.
        Diese Portnummern sind aber elementare Voraussetzung für die internen „Übersetzungen“ eines NAT-Routers.
    \end{itemize}
    \item Übergeordnete Protokolle übertragen die IP als Payload (z.\ B. ftp)
    \begin{itemize}
        \item Begründung: Wenn in den Payload-Daten eine IP übertragen wird, kann der NAT-Übersetzungsprozess nicht stattfinden.
        Ein Rechner in einem privaten Netz würde z.\ B. als Ziel IP 192.168.178.20 in die Payload-Daten kodieren, die auf Empfängerseite (bei einem entfernten Server) nicht verwendet werden kann (weil privat).
    \end{itemize}
    \item Verschlüsselte Verbindungen werden genutzt
    \begin{itemize}
        \item Begründung: Wenn der NAT-Router bereits verschlüsselte IP-Pakete erhält, kann er ggf.\ nicht mehr IP und Portnummer aus den Paketen lesen.
    \end{itemize}
\end{itemize}

\minisec{Was wird bei Port Restricted Cone NAT bei Antwortpaketen überprüft? Was muss sich der NAT-Router dazu merken?}
Dieser Fall erweitert den restricted cone NAT in der Art, dass auch nur der entfernte Port mit dem internen Rechner kommunizieren kann.
Der Router merkt sich Ziel-IP und Ziel-Port.